% Színek
\definecolor{lightblue}{rgb}{0.68, 0.85, 0.9}
\definecolor{apricot}{rgb}{0.98, 0.73, 0.51}
\definecolor{fullblue}{rgb}{0.0, 0.0, 0.99}

% Megjegyzések személyekre
\newcommand{\kovi}[1]{\todo[color=green]{\textbf{Kovi:} #1}}
\newcommand{\koviLine}[1]{\todo[inline,color=green]{#1}}

\newcommand{\szarnyasg}[1]{\todo[color=lightblue]{\textbf{G:} #1}}
\newcommand{\szarnyasgLine}[1]{\todo[inline,color=lightblue]{#1}}

%--------------------------------------------------------------------------------------
% Redefine reference style
%--------------------------------------------------------------------------------------
\newcommand{\figref}[1]{\ref{fig:#1}.}
\renewcommand{\eqref}[1]{(\ref{eq:#1}).}
\newcommand{\listref}[1]{\ref{listing:#1}.}
\newcommand{\secref}[1]{\ref{sec:#1}.}
\newcommand{\tabref}[1]{\ref{tab:#1}.}
\newcommand{\exref}[1]{\ref{example:#1}.}

\newcommand{\afigref}[1]{\aref{fig:#1}.}
\newcommand{\aeqref}[1]{(\aref{eq:#1}).}
\newcommand{\alistref}[1]{\aref{listing:#1}.}
\newcommand{\asecref}[1]{\aref{sec:#1}.}
\newcommand{\atabref}[1]{\aref{tab:#1}.}
\newcommand{\aexref}[1]{\aref{example:#1}.}

\newcommand{\Afigref}[1]{\Aref{fig:#1}.}
\newcommand{\Aeqref}[1]{(\Aref{eq:#1}).}
\newcommand{\Alistref}[1]{\Aref{listing:#1}.}
\newcommand{\Asecref}[1]{\Aref{sec:#1}.}
\newcommand{\Atabref}[1]{\Aref{tab:#1}.}
\newcommand{\Aexref}[1]{\Aref{example:#1}.}
