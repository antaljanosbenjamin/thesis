\newcommand{\assign}{\rightarrow}

\newcommand{\lxor}{\oplus}

\newcommand{\asc}{\uparrow}
\newcommand{\desc}{\downarrow}

\newcommand{\tuple}[1]{\langle #1 \rangle}
\newcommand{\schematuple}[1]{\left\langle #1 \right\rangle}
\newcommand{\concatenation}{\Vert}

\newcommand{\literal}[1]{\mathsf{#1}}
\newcommand{\atom}[1]{\mathsf{#1}}

\newcommand{\colonseparator}{:}

\newcommand{\var}[1]{\mathit{#1}}
%\newcommand{\edgevariable}[2]{\var{#1}\ifstrempty{#2}{}{\hspace{-0.6ex}\colonseparator\hspace{-0.6ex}{\atom{#2}}}}
\newcommand{\edgevariable}[2]{\var{#1}\ifstrempty{#2}{}{\colonseparator{\atom{#2}}}}
\newcommand{\vertexvariable}[2]{(\var{#1}\ifstrempty{#2}{}{\colonseparator{\atom{#2}}})}

% see http://tug.ctan.org/info/symbols/comprehensive/symbols-a4.pdf

%%%%%%%%%%%%%%%%%%%%% operator symbols %%%%%%%%%%%%%%%%%%%%%

% Join symbols in TikZ

% https://tex.stackexchange.com/a/20069
\newcommand{\tikzunit}{1.5ex}

% https://tex.stackexchange.com/a/15037
\FPpow\triangleheight{3}{0.5}
\FPdiv\triangleheight{\triangleheight}{2}

\newtoggle{leftouterjoin}
\newtoggle{semijoin}
\newtoggle{antijoin}

\newcommand{\tikzjointemplate}[1]{%
	{% INTENTIONAL BRACES to limit the scope of toogle settings in #1
	#1
	\text{\,%
		% X axis is scaled with sqrt(3)/2
		\tikz[baseline, x=\tikzunit*\triangleheight, y=\tikzunit, join=round, cap=round]{
			%\draw[help lines] (-2, 0) grid[step=1] (2, 1);
			%
			% left triangle
			\draw (-1,0) -- (0,0.5) -- (-1,1) -- cycle;
			% right triangle without horizontal edge
			\draw (+1,0) -- (0,0.5) -- (+1,1);
			%
			% right horizontal edge
			\ifboolexpr{ not ( togl {semijoin} or togl {antijoin} ) }
			{
				\draw (+1,0) -- (+1,1);
			}{}
			%
			\iftoggle{leftouterjoin}
			{
				\draw (-1,0) -- (-1.5,0);
				\draw (-1,1) -- (-1.5,1);
			}{}
			%
			\iftoggle{antijoin}
			{
				\draw (-1.2,1.2) -- (+1.2,1.2);
			}{}
		}%
\,%
}}}

\newcommand{\joinsymbol}{%
	\tikzjointemplate{}%
}
\newcommand{\antijoinsymbol}{%
	\tikzjointemplate{\toggletrue{antijoin}}%
}
\newcommand{\leftouterjoinsymbol}{%
	\tikzjointemplate{\toggletrue{leftouterjoin}}%
}
\newcommand{\semijoinsymbol}{%
	\tikzjointemplate{\toggletrue{semijoin}}%
}

% based on http://tex.stackexchange.com/questions/20740/symbols-for-outer-joins
%\newcommand{\leftouterjoinwhiskerwidth}{.26em}
%\newcommand{\leftouterjoinwhiskerthickness}{.4pt}

%\def\ojoin{\setbox0=\hbox{$\Join$}\rule[0.1ex]{.27em}{.4pt}\llap{\rule[1.3ex]{.27em}{.4pt}}}

%\def\leftouterjoin{\mathbin{\ojoin\mkern-5.8mu\Join}}
%\def\rightouterjoin{\mathbin{\Join\mkern-5.8mu\ojoin}}
%\def\fullouterjoin{\mathbin{\ojoin\mkern-5.8mu\Join\mkern-5.8mu\ojoin}}

%\def\ojoin{\setbox0=\hbox{$\Join$}\rule[0.1ex]{\leftouterjoinwhiskerwidth}{\leftouterjoinwhiskerthickness}\llap{\rule[1.4ex]{\leftouterjoinwhiskerwidth}{\leftouterjoinwhiskerthickness}}}
%\newcommand{\leftouterjoinsymbol}{\mathbin{\ojoin\mkern-6.4mu\Join}} %bowtie

%\newcommand{\joinsymbol}         {{\tiny \,\textifsym{|><|} \,}}
%\newcommand{\joinsymbol}         {\bowtie}
%\newcommand{\antijoinsymbol}     {{\tiny \,\textifsym{|>}   \,}}
%\newcommand{\leftouterjoinsymbol}{{\tiny \,\textifsym{d|><|b}\,}}
%\newcommand{\semijoinsymbol}     {{\tiny \,\textifsym{|><}  \,}}

%\newcommand{\joinsymbol}{\bowtie}
%\newcommand{\antijoinsymbol}{\triangleright}
%\newcommand{\leftouterjoinsymbol}{{\tiny \,\textifsym{d|><|}\,}}
%\newcommand{\semijoinsymbol}{\ltimes}

%%%%%%%%%%%%%%%%%%%%% operator names %%%%%%%%%%%%%%%%%%%%%

\newcommand{\relalgop}[1]{\textsc{#1}}

\newcommand{\vertexfree}{\medcircle}
\newcommand{\vertexmaybe}{\ocirc}
\newcommand{\vertexbound}{\odot}

\newcommand{\getverticesop}{\vertexfree}

\newcommand{\edgearrow}[1]{\mkern-2.8mu #1 \mkern-2.6mu}

\newcommand{\getedgesopdirected}  {{\getverticesop}{\edgearrow{\rightarrow}    }{\getverticesop}}
\newcommand{\getedgesopundirected}{{\getverticesop}{\edgearrow{\leftrightarrow}}{\getverticesop}}

\newcommand{\expandbothop}{{\vertexbound}{\edgearrow{\leftrightarrow}}{\vertexmaybe}}
\newcommand{\expandoutop} {{\vertexbound}{\edgearrow{\rightarrow}    }{\vertexmaybe}}
\newcommand{\expandinop}  {{\vertexbound}{\edgearrow{\leftarrow}     }{\vertexmaybe}}

\newcommand{\alldifferentop}{\not\equiv}
\newcommand{\duplicateeliminationop}{\delta}
\newcommand{\sortop}{\tau}
\newcommand{\sortandtopop}{\topop\sortop}
\newcommand{\productionop}{\Omega}
\newcommand{\projectionop}{\pi}
\newcommand{\renameop}{\rho}

\newcommand{\createop}{\zeta}
\newcommand{\mergeop}{\zeta^\star}
\newcommand{\deleteop}{\chi}
\newcommand{\setop}{set}
\newcommand{\removeop}{remove}
\newcommand{\selectionop}{\sigma}
\newcommand{\groupingop}{\gamma}
\newcommand{\topop}{\lambda}
\newcommand{\unwindop}{\omega}
\newcommand{\unnestop}{\mu}
\newcommand{\nestop}{\nu}

\newcommand{\joinop}{\joinsymbol}
\newcommand{\transitivejoinop}{\joinop_{\ast}}
\newcommand{\antijoinop}{\antijoinsymbol}
\newcommand{\semijoinop}{\semijoinsymbol}
\newcommand{\leftouterjoinop}{\leftouterjoinsymbol}
\newcommand{\thetaleftouterjoinop}{\leftouterjoinsymbol_\theta}
\newcommand{\unionop}{\cup}
\newcommand{\Unionop}{\bigcup}
\newcommand{\bagunionop}{\uplus}
\newcommand{\Bagunionop}{\biguplus}
\newcommand{\minusop}{\setminus}
\newcommand{\intersectionop}{\cap}
\newcommand{\cartesianproductop}{\times}

%%%%%%%%%%%%%%%%%%%%% operator definitions %%%%%%%%%%%%%%%%%%%%%

%%%%%%%%%% nullary operators %%%%%%%%%%

\makeatletter
\newlength{\negph@wd}
\DeclareRobustCommand{\negphantom}[1]{%
  \ifmmode
    \mathpalette\negph@math{#1}%
  \else
    \negph@do{#1}%
  \fi
}
\newcommand{\negph@math}[2]{\negph@do{$\m@th#1#2$}}
\newcommand{\negph@do}[1]{%
  \settowidth{\negph@wd}{#1}%
  \hspace*{-\negph@wd}%
}

\newcommand{\getvertices}[2]{\getverticesop_{\var{#1}}^{\atom{#2}}}

\newcommand{\getedges}[7]{\tensor*[^{\atom{#2}}_{\var{#1}}]{{\vertexfree}{\edgearrow{#7[\var{#5}]{\atom{#6}}}}{\vertexfree}}{^{\atom{#4}}_{\var{#3}}}}
\newcommand{\getedgesdirected}[6]{\getedges{#1}{#2}{#3}{#4}{#5}{#6}{\xrightarrow}}
\newcommand{\getedgesundirected}[6]{\getedges{#1}{#2}{#3}{#4}{#5}{#6}{\xleftrightarrow}}

%\newcommand{\dual}{\var{Dual}}
\newcommand{\dual}{\{\tuple{}\}}

%%%%%%%%%% unary operators %%%%%%%%%%

% expand operators
\newcommand{\kleenestar}{\ast}
\newcommand{\nagivationbody}[3]{\,_{\vertexvariable{#1}{}}^{\vertexvariable{#2}{#3}}}
\newcommand{\expandedgevariable}[3]{
  {%\scriptstyle
	%\left[
	% #3: minHops, cannot be empty
	% #4: maxHops, if empty, default to infinity
	\atom{#1}
	\ifstrequal{#2}{1} % minHops = 1
	{
		\ifstrequal{#3}{1}
		{} % minHops = 1 and maxHops = 1 -> write nothing
		{\kleenestar_\atom{#2}^\atom{#3}} % minHops = 1 and maxHops != 1
	} % minHops != 1
	{\kleenestar_\atom{#2}^\atom{#3}}
	%\right]
  }}


%%%%%%%%%%%%%%%%%%%%%%%%%%%%%%%%%%%%%%%%%%%%%%%%%%%%%%%%%%%%%%%%%%%%%%%%%%%%%%%%
\newcommand{\expand}[8]{{}_{\var{#1}} {\vertexbound}{\edgearrow{#8[\var{#4}]{\expandedgevariable{#5}{#6}{#7}}}}{\vertexfree}_\var{#2}^\atom{#3}}

\newcommand{\expandboth}[7]{\expand{#1}{#2}{#3}{#4}{#5}{#6}{#7}{\xleftrightarrow}}
\newcommand{\expandout }[7]{\expand{#1}{#2}{#3}{#4}{#5}{#6}{#7}{\xrightarrow}    }
\newcommand{\expandin  }[7]{\expand{#1}{#2}{#3}{#4}{#5}{#6}{#7}{\xleftarrow}     }

\newcommand{\expandintomaybe}[8]{{}_{\var{#1}} {\vertexbound}{\edgearrow{#8[\var{#4}]{\expandedgevariable{#5}{#6}{#7}}}}{\vertexmaybe}_\var{#2}^\atom{#3}}

\newcommand{\expandbothintomaybe}[7]{\expandintomaybe{#1}{#2}{#3}{#4}{#5}{#6}{#7}{\xleftrightarrow}}
\newcommand{\expandoutintomaybe }[7]{\expandintomaybe{#1}{#2}{#3}{#4}{#5}{#6}{#7}{\xrightarrow}    }
\newcommand{\expandinintomaybe  }[7]{\expandintomaybe{#1}{#2}{#3}{#4}{#5}{#6}{#7}{\xleftarrow}     }

\newcommand{\expandintobound}[8]{{}_{\var{#1}} {\vertexbound}{\edgearrow{#8[\var{#4}]{\expandedgevariable{#5}{#6}{#7}}}}{\vertexbound}_\var{#2}^\atom{#3}}

\newcommand{\expandbothintobound}[7]{\expandintobound{#1}{#2}{#3}{#4}{#5}{#6}{#7}{\xleftrightarrow}}
\newcommand{\expandoutintobound }[7]{\expandintobound{#1}{#2}{#3}{#4}{#5}{#6}{#7}{\xrightarrow}    }
\newcommand{\expandinintobound  }[7]{\expandintobound{#1}{#2}{#3}{#4}{#5}{#6}{#7}{\xleftarrow}     }

\newcommand{\unwind}[1]{\unwindop_{#1}}
\newcommand{\unwindarrow}{\Rightarrow}
\newcommand{\alldifferent}[1]{\alldifferentop_{#1}}
\newcommand{\duplicateelimination}{\duplicateeliminationop}
\newcommand{\sort}[1]{\sortop_{#1}}
\newcommand{\projection}[1]{\projectionop_{#1}}

\newcommand{\topp}[2]{\topop_{#1}^{#2}}
\newcommand{\sortandtop}[3]{\left\{ \sortop_{#1} \topop_{#2}^{#3} \right\}}
\newcommand{\skipp}[1]{\topop^{#1}}
\newcommand{\limit}[1]{\topop_{#1}}
% CUD operators
\newcommand{\create}[1]{\createop_{#1}}
\newcommand{\mergenode}[1]{\mergeop_{#1}}
% delete: #1: '*' in case of DETACH DELETE, otherwise empty. #2: variable list to delete
\newcommand{\delete}[2]{\deleteop_{#2}^{#1}}
\newcommand{\setnode}[1]{\setop_{#1}}
\newcommand{\removenode}[1]{\removeop_{#1}}

\newcommand{\selection}[1]{\selectionop_{#1}}
\newcommand{\rename}[1]{\renameop_{#1}}
\newcommand{\grouping}[2]{\groupingop_{#1}^{#2}}
\newcommand{\production}[1]{\productionop_{#1}}
\newcommand{\nest}[1]{\nestop_{#1}}
\newcommand{\unnest}[1]{\unnestop_{#1}}

%%%%%%%%%% binary operators %%%%%%%%%%

\newcommand{\join}{\joinop}
%\newcommand{\transitivejoin}[2]{{\joinop}{ \text{\scriptsize{\(\ast_{#1}^{#2}\)}} }}
%\newcommand{\transitivejoin}[2]{{\joinop}{\ast_{#1}^{#2}}}
\newcommand{\transitivejoin}[2]{\overset{\ast_{#1}^{#2}}{\joinop}}
\newcommand{\antijoin}{\antijoinop}
\newcommand{\semijoin}{\semijoinop}
\newcommand{\leftouterjoin}{\leftouterjoinop}
\newcommand{\thetaleftouterjoin}{\thetaleftouterjoinop}
\newcommand{\union}{\unionop}
\newcommand{\Union}{\Unionop}
\newcommand{\bagunion}{\bagunionop}
\newcommand{\Bagunion}{\bagunionop}
%\newcommand{\minus}{\minusop}
\newcommand{\intersection}{\intersectionop}
\newcommand{\cartesianproduct}{\cartesianproduct}

% colors

\definecolor{red}{HTML}{d7191c}
\definecolor{lilac}{HTML}{984ea3}
\definecolor{red}{HTML}{d7191c}
\definecolor{blue}{HTML}{2b83ba}

\colorlet{nestedschemacolor}{blue}
\colorlet{flatschemacolor}{red}
\colorlet{nullarynodecolor}{lilac}

\newcommand{\operatortext}[1]{#1\xspace}

% nullary
\newcommand{\getverticestext}{\operatortext{get-vertices}}
\newcommand{\getedgestext}{\operatortext{get-edges}}

% unary
\newcommand{\expandtext}{\operatortext{expand}}
\newcommand{\expandouttext}{\operatortext{expand-out}}
\newcommand{\expandintext}{\operatortext{expand-in}}
\newcommand{\expandbothtext}{\operatortext{expand-both}}

\newcommand{\transitiveexpandtext}{\operatortext{transitive expand}}
\newcommand{\transitiveexpandouttext}{\operatortext{transitive expand-out}}
\newcommand{\transitiveexpandintext}{\operatortext{transitive expand-in}}
\newcommand{\transitiveexpandbothtext}{\operatortext{transitive expand-both}}

\newcommand{\projectiontext}{\operatortext{projection}}
\newcommand{\renametext}{\operatortext{rename}}
\newcommand{\selectiontext}{\operatortext{selection}}
\newcommand{\alldifferenttext}{\operatortext{all-different}}
\newcommand{\duplicateeliminationtext}{\operatortext{duplicate-elimination}}
\newcommand{\groupingtext}{\operatortext{grouping}}
\newcommand{\sorttext}{\operatortext{sort}}
\newcommand{\toptext}{\operatortext{top}}
\newcommand{\sortandtoptext}{\operatortext{sort-and-top}}
\newcommand{\unwindtext}{\operatortext{unwind}}
\newcommand{\createtext}{\operatortext{create}}
\newcommand{\deletetext}{\operatortext{delete}}
\newcommand{\mergetext}{\operatortext{merge}}
\newcommand{\removetext}{\operatortext{remove}}
\newcommand{\settext}{\operatortext{set}}
\newcommand{\productiontext}{\operatortext{production}}

% binary
\newcommand{\baguniontext}{\operatortext{bag union}}
\newcommand{\antijointext}{\operatortext{antijoin}}
\newcommand{\semijointext}{\operatortext{semijoin}}
\newcommand{\cartesianproducttext}{\operatortext{Cartesian product}}
\newcommand{\transitivejointext}{\operatortext{transitive join}}
\newcommand{\leftouterjointext}{\operatortext{left outer join}}
\newcommand{\thetaleftouterjointext}{\operatortext{theta left outer join}}
\newcommand{\jointext}{\operatortext{natural join}}
\newcommand{\uniontext}{\operatortext{union}}

\newcommand{\relnull}{\ensuremath{\mathsf{NULL}}}
\newcommand{\relnulls}{\ensuremath{\tuple{\mathsf{NULL}}}}


% misc
\newcommand{\schema}[1]{\mathrm{sch} \left( #1 \right)}
\newcommand{\schemabrackets}[1]{\mathrm{sch}\left[ #1 \right]}

\newcommand{\labelsLlong}{l_1, \ldots, l_\var{k}}
\newcommand{\labelsL}{L}

\newcommand{\labelsAlong}{l_{1,1}, \ldots, l_{1,\var{m}}}
\newcommand{\labelsA}{L1}

\newcommand{\labelsBlong}{l_{2,1}, \ldots, l_{2,\var{n}}}
\newcommand{\labelsB}{L2}

\newcommand{\typeslong}{t_1, \ldots, t_\var{n}}
\newcommand{\types}{T}

\newcommand{\alias}{/}