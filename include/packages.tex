% thanks to http://tex.stackexchange.com/a/47579/71109
\usepackage{t1enc}
\usepackage{ifxetex}
\usepackage{ifluatex}
\newif\ifxetexorluatex % a new conditional starts as false
\ifnum 0\ifxetex 1\fi\ifluatex 1\fi>0
   \xetexorluatextrue
\fi

\ifxetexorluatex
  \usepackage{fontspec}
\else
  \usepackage[T1]{fontenc}
  \usepackage[utf8]{inputenc}
  \usepackage[lighttt]{lmodern}
\fi

\usepackage[english,magyar]{babel} % Alapértelmezés szerint utoljára definiált nyelv lesz aktív, de később külön beállítjuk az aktív nyelvet.

%\usepackage{cmap}
\usepackage{amsfonts,amsmath,amssymb} % Mathematical symbols.
\usepackage[ruled,boxed,resetcount,linesnumbered,]{algorithm2e}
\renewcommand*{\algorithmcfname}{Algoritmus}
\SetKwInput{KwData}{Input}
\SetKwInput{KwResult}{Output}
%\SetKwInput{While}{amíg}
%\SetKwInput{If}{ha}
%\SetKwInput{Then}{akkor}
%\SetKwInput{Else}{különben}

 % For pseudocodes. % beware: this is not compatible with LuaLaTeX, see http://tex.stackexchange.com/questions/34814/lualatex-and-algorithm2e
\usepackage{booktabs} % For publication quality tables for LaTeX
\usepackage{graphicx}

%\usepackage{fancyhdr}
%\usepackage{lastpage}

\usepackage{anysize}
%\usepackage{sectsty}
\usepackage{setspace} % For setting line spacing

\usepackage[unicode]{hyperref} % For hyperlinks in the generated document.
\usepackage{xcolor}
\usepackage{listings} % For source code snippets.

\usepackage[amsmath,thmmarks]{ntheorem} % Theorem-like environments.

\usepackage[hang]{caption}
\usepackage{subcaption}

\singlespacing

\newcommand{\selecthungarian}{
	\selectlanguage{magyar}
	\setlength{\parindent}{2em}
	\setlength{\parskip}{0em}
	\frenchspacing
}

\newcommand{\selectenglish}{
	\selectlanguage{english}
	\setlength{\parindent}{0em}
	\setlength{\parskip}{0.5em}
	\nonfrenchspacing
	\renewcommand{\figureautorefname}{Figure}
	\renewcommand{\tableautorefname}{Table}
	\renewcommand{\partautorefname}{Part}
	\renewcommand{\chapterautorefname}{Chapter}
	\renewcommand{\sectionautorefname}{Section}
	\renewcommand{\subsectionautorefname}{Section}
	\renewcommand{\subsubsectionautorefname}{Section}
}

% megjegyzésekhez
\newcommand{\todo}[2][]{}
%\usepackage{todonotes}

% G cikkének képleteihez
% thanks to http://tex.stackexchange.com/a/47579/71109
\usepackage{ifxetex}
\usepackage{ifluatex}
\newif\ifxetexorluatex % a new conditional starts as false
\ifnum 0\ifxetex 1\fi\ifluatex 1\fi>0
   \xetexorluatextrue
\fi

\ifxetexorluatex
  \usepackage{fontspec}
\else
  \usepackage[T1]{fontenc}
  \usepackage[utf8]{inputenc}
%  \usepackage[lighttt]{lmodern}
\fi

\usepackage{amsmath}
\usepackage{amssymb}
\usepackage{etoolbox}
\usepackage{xspace}
\newtoggle{textualoperators}

\usepackage{tikz}
% Using the forest package, there is a known issue that edges sometimes cut through child nodes.
% See the package documentation on CTAN (https://www.ctan.org/pkg/forest), section 6.2. "Known bugs", "Edges cutting through sibling nodes":
% While it would be possible to fix the situation after child alignment (at least for some child alignment methods), I have decided against that, since the distances between siblings would soon become too large. [...] The bottomline is, please use manual adjustment to fix such situations.
\usepackage{forest}

\usetikzlibrary{arrows,shapes.arrows,automata}

\tikzset{every node/.style={draw}}

\usepackage{tensor}

\usepackage{fp}

% \dashuline-hoz
\usepackage[normalem]{ulem}
% középre
\usepackage{adjustbox}

\usepackage{algorithmic}
\usepackage{csquotes}
\usepackage{multirow}
\usepackage{array}
\usepackage{autobreak}
\usepackage{arydshln}
\usepackage{mathtools}
\usepackage{ifsym}
\usepackage{tabularx}
\usepackage{MnSymbol}
\usepackage{rotating}
%\usepackage{fixltx2e}
\usepackage{colortbl}
\usepackage{enumitem}

\usepackage[numbers]{natbib}
\usepackage{makecell}